% Release 0.4alpha (2021-11-01) of xintsession.tex
% author: Jean-Francois Burnol
% License: LPPL 1.3c (author-maintained)
% Usage: execute "etex (or pdftex) xintsession"
% If available use "rlwrap etex xintsession" for a better experience
\edef\xsrestorenewlinechar{\newlinechar\the\newlinechar\relax}\newlinechar10\relax%
% we load it with \newlinechar set in order for its error messages
% to be nicer (since xintexpr 1.4g)
\input xintexpr.sty\relax%
\unless\ifdefined\xintreloadxintlog% defined at 1.4e
{\catcode32=11%
\errhelp{Please update xintexpr to at least 1.4e.  Aborting input.^^JSay `x' to exit.}%
\errmessage{xintexpr is too old.  Say `x' to exit, `h' for help}}\endinput\fi%
\input xintbinhex.sty\relax%
\input polexpr.sty\relax% this will crash with polexpr versions < 0.8
\let\PolToExprOneTerm\PolToExprOneTermStyleB%
% \XINTrestorecatcodes.  Although we morally assume normal catcodes
% we use it as a convenience: strangely named, it is to be used in
% an \edef to store there catcodes and \endlinechar (but not
% \newlinechar).  It was renamed at 1.4e and this is one reason
% why we aborted input with earlier xintexpr...
\edef\xsrestorecatcodes{\XINTrestorecatcodes\xsrestorenewlinechar}%
\endlinechar 13 % default anyhow
\newlinechar 10
\catcode`_ 11
\catcode`& 11
\catcode`@ 11
\catcode`^ 11 
\catcode`$ 7 %$
\catcode0 12
\newwrite\xs_out
\edef\xs_name{\jobname-\expandafter\xint_gobble_ii\the\year
                      \expandafter\xint_gobble_i\the\numexpr100+\month\relax
                      \expandafter\xint_gobble_i\the\numexpr100+\day\relax
                      _%
                      \expandafter\xint_gobble_i\the\numexpr(\time+30)/60+99h%
                      \expandafter\xint_gobble_i\the\numexpr100+\time-60*((\time+30)/60-1).tex}%
\immediate\openout\xs_out=\xs_name\relax
\begingroup\catcode37 12
\long\gdef\xs_message#1{\message{#1$$J}\immediate\write\xs_out{%#1}}\endgroup%$$
{\catcode13\active\def
{$$J}\catcode32\active\let \space%$$
\catcode`?=0\catcode`\\=12\relax%
?xdef?xs_logo{%
               _____       _____
        ____  ____(_)________  /_________  _________________
        __  |/_/_  /__  __ \  __/  _ \_  |/_/__  __ \_  ___/
        __>  < _  / _  / / / /_ /  __/_>  < __  /_/ /  /
        /_/|_| /_/  /_/ /_/\__/ \___//_/|_| _  .___//_/
                                              /_/
_____       _____                         __________            ______
___(_)________  /__________________ ________  /___(_)__   ______   / /
__  /__  __ \  __/  _ \_  ___/  __ `/  ___/  __/_  /__ | / /  _ \ /_/
_  / _  / / / /_ /  __/  /   / /_/ // /__ / /_ _  / __ |/ //  __/__
/_/  /_/ /_/\__/ \___//_/    \__,_/ \___/ \__/ /_/  _____/ \___//_/%
}%
?xdef?xs_copyright{%
  xintsession 0.4alpha (2021-11-01)
  Copyright (c) Jean-Francois Burnol, 2021.%
}}%
\def\xs_no{0}%
\XINT_expr_defvar_one{@___}{{0}}%
\XINT_expr_defvar_one{@__}{{0}}%
\XINT_expr_defvar_one{@_}{{0}}%
\def\xs_prime_invite{>>> }%
\def\xs_second_invite{... }%
\def\xs_fetch
{%
  \let\xs_in\empty
  \let\xs_invite\xs_prime_invite
  \xs_fetch_aa
}%
\def\xs_fetch_aa
{%
  \message{\xs_invite}\read-1to\xs_buf
  \expandafter\xs_fetch_a\xs_buf\xs_fetch_a
}%
\long\def\xs_fetch_a#1\xs_fetch_a
{%
  \odef\xs_firstitem{\romannumeral0\xintfirstitem{#1.}}%
  \ifcat\relax\expandafter\noexpand\xs_firstitem
    \expandafter\xs_cs
  \else
   \if &\expandafter\noexpand\xs_firstitem
    \expandafter\expandafter\expandafter\xs_magic
   \fi
  \fi
  \if;\expandafter\noexpand\romannumeral0\xintlastitem{#1}%
    \xs_fetch_b#1^%
  \else
    \odef\xs_in{\xs_in#1}%
    \let\xs_invite\xs_second_invite
    \expandafter\xs_fetch_aa
  \fi%
}%
\long\def\xs_cs\if;\expandafter\noexpand\romannumeral0\xintlastitem#1#2\fi
{%
   \xs_message{(executing \detokenize{#1} in background)}%
% potential problems if a group is closed by #1.
% should I issue \xs_resetenv here ?
   #1%
   \message{$$J}\xs_fetch_aa%$$
}%
\def\xs_firstfour#1#2#3#4#5\xint_bye{\edef\xs_magic_in{#1#2#3#4}\def\xs_magic_inb{#5}}%
\long\def\xs_magic\if;\expandafter\noexpand\romannumeral0\xintlastitem#1#2\fi
{%
  \xs_firstfour#1\empty\empty\empty\empty\xint_bye
  \ifcsname xs_magic@\xs_magic_in\endcsname
      \csname xs_magic@\xs_magic_in\expandafter\endcsname
  \else
   \xs_message{(\detokenize\expandafter{\xint_gobble_i#1} is not a magic keyword:
            ignored)$$J}%$$
  \fi
  \xs_fetch_aa
}%
\long\def\xs_fetch_b#1;^%
{%
   \odef\xs_in{\xs_in#1}\expandafter\xs_fetch_c
}%
\def\xs_fetch_c
{%
   \if\relax\romannumeral`$$@\detokenize\expandafter{\xs_in}\relax\expandafter\xs_pause\fi%$$
   \expandafter\xs_parse_a\xs_in:=^%
}%
\def\xs_pause\expandafter\xs_parse_a\xs_in:=^{\xs_pause_a\xs_fetch_aa}%
\def\xs_pause_a\xs_fetch_aa
{%
   \xs_message{Need a break? OK, use \string\xintsession\space if you change your mind.}%
   \xs_resetenv
}%
\def\xs_bye\xs_fetch_aa
{%
   \xs_message{Did I say something wrong?}%
   \xs_message{Session transcript written on \xs_name}%
%   \xs_resetenv
   \immediate\closeout\xs_out
% expeditive exit because I need to think about generating new output
% filename with a check that it is not same (a minute may be too coarse?).
% Add random postfix? or simply do not bother rather...
   \csname bye\endcsname
}%
\def\xs_parse_a#1:=#2%
{%
   \xint_gob_til_^#2\xs_calc ^%
   \def\xs_left{#1}\xs_parse_b#2%
}%
\def\xs_calc ^#1^%
{%
   \edef\xs_no{\the\numexpr1+\xs_no}%
   \xs_xintdefvar @_\xs_no=\xs_in;%
   \immediate\write\xs_out
   {\expandafter\string\xs_xintdefvar\space @_\xs_no:=%
                \detokenize\expandafter{\xs_in};}%
   \XINT_expr_defvar_one{@___}{\XINT_expr_varvalue_@__}%
   \XINT_expr_defvar_one{@__}{\XINT_expr_varvalue_@_}%
   \XINT_expr_defvar_one{@_}{\csname XINT_expr_varvalue_@_\xs_no\endcsname}%
   \xs_message{%
     @_\expandafter\xs_keepsix\xs_no\space\space\space\space\space\relax
     \xs_xinteval{@_\xs_no}%
   }%
   \xs_fetch
}%
\def\xs_keepsix#1#2#3#4#5#6#7\relax{#1#2#3#4#5#6}%
\def\xs_parse_b #1:=^%
{%
   \def\xs_right{#1}\expandafter\xs_parse_c\xs_left(;%
}%
\def\xs_parse_c #1(#2%
{%
   \edef\xs_funcname{\expandafter\xint_zapspaces\detokenize{#1} \xint_gobble_i}%
   \xint_gob_til_sc#2\xs_defvar;\xs_deffunc
}%
\edef\xs_fourspaces{\space\space\space\space}%
\def\xs_deffunc #1;%
{%
% attention: \xs_xintdeffunc needs to see parentheses immediately
   \expandafter\xs_xintdeffunc\xs_left=\xs_right;%
   \immediate\write\xs_out
   {\expandafter\string\xs_xintdeffunc\space\detokenize\expandafter{\xs_left}:=%
                \detokenize\expandafter{\xs_right};}%
   \expandafter\ifx\xs_xintdeffunc\poldef
        \xs_message{\xs_funcname\space = \PolToExpr{\xs_funcname}}%
        \xs_message{\xs_fourspaces -->
                    &GenFloat(\xs_funcname) lets
                    \xs_funcname\space become usable as function in fp mode}%
        \xs_message{\xs_fourspaces
                    --> &ROOTS(\xs_funcname) (resp. &ROOTS(\xs_funcname,N))
                       finds all rational roots exactly and}%
        \xs_message{\xs_fourspaces\xs_fourspaces
                    all irrational roots with at least
                    \xintsessiondefaultrootsprecision\space(resp. N)
                    fractional digits}%
   \else\xs_message{Function \xs_funcname\space defined}%
   \fi\xs_fetch
}%
\def\xs_defvar;\xs_deffunc
{%
   \edef\xs_no{\the\numexpr1+\xs_no}%
   \xs_xintdefvar @_\xs_no=\xs_right;%
   \immediate\write\xs_out
   {\expandafter\string\xs_xintdefvar\space\detokenize\expandafter{\xs_left}:=%
                \detokenize\expandafter{\xs_right};}%
   \XINT_expr_defvar_one{@___}{\XINT_expr_varvalue_@__}%
   \XINT_expr_defvar_one{@__}{\XINT_expr_varvalue_@_}%
   \XINT_expr_defvar_one{@_}{\csname XINT_expr_varvalue_@_\xs_no\endcsname}%
   % this is to allow multiple assignments
   \expandafter\xs_xintdefvar\xs_left=@_\xs_no;%
   \xs_message{%
     @_\expandafter\xs_keepsix\xs_no\space\space\space\space\space\relax
     \xs_xinteval{@_\xs_no}%
   }%
   \xs_fetch
}%
\let\xs_magic@&pau\xs_pause_a
\let\xs_magic@&bye\xs_bye
\def\xs_magic@&fp
{%
   \xs_message{fp mode (\xinttheDigits\space digits)}%
   \def\xs_mode{fp (Digits=\xinttheDigits)}%
   \def\xs_xintdefvar{\xintdeffloatvar}%
   \def\xs_xintdeffunc{\xintdeffloatfunc}%
   \def\xs_xinteval{\xintfloateval}%
}%
\expandafter\def\csname xs_magic@&fp=\endcsname
{%
   \xintDigits*:=\xs_magic_inb\relax
   \xs_message{fp mode (log and trig reloaded at Digits=\xinttheDigits)}%
   \def\xs_mode{fp=\xinttheDigits}%
   \def\xs_xintdefvar{\xintdeffloatvar}%
   \def\xs_xintdeffunc{\xintdeffloatfunc}%
   \def\xs_xinteval{\xintfloateval}%
}%
\def\xs_magic@&exa
{%
   \xs_message{exact mode (floating point evaluations use \xinttheDigits\space digits)}%
   \def\xs_mode{exact}%
   \def\xs_xintdefvar{\xintdefvar}%
   \def\xs_xintdeffunc{\xintdeffunc}%
   \def\xs_xinteval{\xinteval}%
}%
\def\xs_magic@&int
{%
   \xs_message{int mode (integer only: / rounds, // is floored division, /: is
   modulo)}%
   \def\xs_mode{int}%
   \def\xs_xintdefvar{\xintdefiivar}%
   \def\xs_xintdeffunc{\xintdefiifunc}%
   \def\xs_xinteval{\xintiieval}%
}%
\def\xs_magic@&pol
{%
   \xs_message{pol mode (i.e. function definitions use \string\poldef)}%
   \def\xs_mode{pol}%
   \def\xs_xintdefvar{\xintdefvar}%
   \def\xs_xintdeffunc{\poldef}%
   \def\xs_xinteval{\xinteval}%
}%
\def\xs_magic@&mod{\xs_message{current mode: \xs_mode}}%
\def\xs_getpolname#1(#2#3)#4\xint_bye{\edef\xs_polname{\xint_zapspaces #2#3 \xint_gobble_i}}%
\def\xs_magic@&Gen
{%
   \expandafter\xs_getpolname\xs_magic_inb\xint_bye
   \PolGenFloatVariant{\xs_polname}%
   \immediate\write\xs_out{\string\PolGenFloatVariant{\xs_polname}}%
   \xs_message{Polynomial function \xs_polname\space for &fp mode usage now defined}%
}%
\def\xintsessiondefaultrootsprecision{10}%
\def\xs_getpolandE#1(#2#3)#4\xint_bye
{%
   \xs_getpolandE_i#2#3,,\xint_bye
}%
\def\xs_getpolandE_i#1,#2,#3\xint_bye
{%
   \edef\xs_polname{\xint_zapspaces #1 \xint_gobble_i}%
   \edef\xs_rootsprecision{\if\relax#3\relax
                           \xintsessiondefaultrootsprecision\else
        \xint_zapspaces #2 \xint_gobble_i\fi}%
}%
\expandafter\def\csname xs_magic@&ROO\endcsname
{%
   \expandafter\xs_getpolandE\xs_magic_inb\xint_bye
   \xs_message{Solving for real roots of \xs_polname\space and assigning them (please wait...)}%
   \PolToSturm{\xs_polname}{@\xs_polname}%
   \PolSturmIsolateZerosGetMultiplicitiesAndRationalRoots{@\xs_polname}%
   \PolEnsureIntervalLengths{@\xs_polname}{-\xs_rootsprecision}%
     \immediate\write\xs_out{\string\PolToSturm{\xs_polname}{@\xs_polname}}%
     \immediate\write\xs_out{\string\PolSturmIsolateZerosGetMultiplicitiesAndRationalRoots{@\xs_polname}}%
     \immediate\write\xs_out{\string\PolEnsureIntervalLengths{@\xs_polname}{-\xs_rootsprecision}}%
     \immediate\write\xs_out{\string\PolPrintIntervals{@\xs_polname}}%
   \xs_polprintintervals
   \xs_message{Square-free irrational part: \PolToExpr{@\xs_polname _sqf_norr}}%
   \ifnum\PolSturmNbOfIsolatedZeros{@\xs_polname}>\PolSturmNbOfRationalRoots{@\xs_polname}
     \xs_message{\xs_fourspaces -->
                 &REFINEROOTS(\xs_polname,N) to extend real irr. roots to N fractional digits}%
   \else
    \xs_message{No irrational real roots.}%
   \fi
}%
\expandafter\def\csname xs_magic@&REF\endcsname
{%
   \expandafter\xs_getpolandE\xs_magic_inb\xint_bye
   \xs_message{Refining real roots of \xs_polname\space to \xs_rootsprecision\space
   digits (please wait...)}%
   \PolEnsureIntervalLengths{@\xs_polname}{-\xs_rootsprecision}%
     \immediate\write\xs_out{\string\PolEnsureIntervalLengths{@\xs_polname}{-\xs_rootsprecision}}%
     \immediate\write\xs_out{\string\PolPrintIntervals{@\xs_polname}}%
   \xs_polprintintervals
}%
\def\xs_polprintintervals
{%
   \begingroup\xintglobaldefstrue
   \def\PolPrintIntervalsNoRealRoots{\xs_message{No real roots.}}%
   \let\PolPrintIntervalsBeginEnv\empty
   \let\PolPrintIntervalsEndEnv\empty
   \def\PolPrintIntervalsKnownRoot{\PolPrintIntervalsPrintExactZero}%
   \def\PolPrintIntervalsUnknownRoot{\xintifSgn{\PolPrintIntervalsTheLeftEndPoint}%
          {\PolPrintIntervalsPrintRightEndPoint}%
          {\PolPrintIntervalsPrintLeftEndPoint}%
          {\PolPrintIntervalsPrintLeftEndPoint}%
       }%
   \def\PolPrintIntervalsPrintExactZero
     {\xs_message{(mult. \PolPrintIntervalsTheMultiplicity)
                  Root\xs_polname _\PolPrintIntervalsTheIndex\space = \PolPrintIntervalsTheLeftEndPoint}%
      \immediate\write\xs_out{\string\xintdefvar\space
         Root\xs_polname _\PolPrintIntervalsTheIndex:=\PolPrintIntervalsTheLeftEndPoint;}%
      \XINT_expr_defvar_one{Root\xs_polname _\PolPrintIntervalsTheIndex}%
         {{\xintRaw{\PolPrintIntervalsTheLeftEndPoint}}}%
     }%
   \def\PolPrintIntervalsPrintLeftEndPoint
     {\xs_message{(mult. \PolPrintIntervalsTheMultiplicity)
                  Root\xs_polname _\PolPrintIntervalsTheIndex\space = \PolPrintIntervalsTheLeftEndPoint...}%
      \immediate\write\xs_out{\string\xintdefvar\space
         Root\xs_polname _\PolPrintIntervalsTheIndex:=\PolPrintIntervalsTheLeftEndPoint;}%
      \XINT_expr_defvar_one{Root\xs_polname _\PolPrintIntervalsTheIndex}%
         {{\xintRaw{\PolPrintIntervalsTheLeftEndPoint}}}%
     }%
   \def\PolPrintIntervalsPrintRightEndPoint
     {\xs_message{(mult. \PolPrintIntervalsTheMultiplicity)
                  Root\xs_polname _\PolPrintIntervalsTheIndex\space = \PolPrintIntervalsTheRightEndPoint...}%
      \immediate\write\xs_out{\string\xintdefvar\space
         Root\xs_polname _\PolPrintIntervalsTheIndex:=\PolPrintIntervalsTheRightEndPoint;}%
      \XINT_expr_defvar_one{Root\xs_polname _\PolPrintIntervalsTheIndex}%
         {{\xintRaw{\PolPrintIntervalsTheRightEndPoint}}}%
     }%
   \def\POL@PrintIntervals@Loop{%
      \POL@SturmIfZeroExactlyKnown\PolPrintIntervalsTheSturmName\PolPrintIntervalsTheIndex
          \PolPrintIntervalsKnownRoot
          \PolPrintIntervalsUnknownRoot
      \xdef\PolPrintIntervalsTheIndex{\the\numexpr\PolPrintIntervalsTheIndex+\@ne}%
        \unless\ifnum\PolPrintIntervalsTheIndex>
               \@nameuse{POL_ZL\PolPrintIntervalsTheSturmName*0}
        \POL@PrintIntervals@DoDefs
        \expandafter\POL@PrintIntervals@Loop
      \fi
   }%
   \PolPrintIntervals{@\xs_polname}%
   \endgroup       
}%
\def\xs_setenv
{%
    \newlinechar10\endlinechar-1\errorcontextlines0\relax
}%
\def\xs_storeenv{\edef\xs_resetenv
{%
    \newlinechar\the\newlinechar
    \endlinechar\the\endlinechar
    \errorcontextlines\the\errorcontextlines
}}%
\def\xintsession
   {\let\xintsession\xintsessionresume
    \xs_setenv\xint_welcome\message{Starting in}\xs_magic@&exa\xs_fetch}%
\def\xintsessionresume
   {\xs_storeenv\xs_setenv\xs_welcome_back\xs_fetch}%
\begingroup
% some things here are weird but this was initially to make the 0.3 transition
% to usage of \read-1 in place of active ^^M with minimal collateral changes
% the active ^^M here is now reduced to avoid to have to type explicit ^^J in the
% definition of the help messages
\catcode13\active\catcode32\active%
\let \space% this is Plain default anyhow
\def
{$$J}%$$
\xdef\xint_welcome{\noexpand\message{\xs_copyright
  Welcome to
\xs_logo

  - input is `;'-terminated and can extend over multiple lines
  - definitions of variables and functions are done with `:='.

  \string\jobname is \jobname
  Transcript will go to log and to \xs_name
  Typing at the prompt
  - `&pause' or `;' exits to normal TeX (typesetting) interaction,
  - `&bye' terminates the session,
  - `&help' triggers help panels,
  - `&exact', `&fp', `&int', or `&pol' switches among modes.
}}%
\xdef\xs_welcome_back{\noexpand\message{%
  You are back to the xintexpr interactive session!
  (current mode: \noexpand\xs_mode, with Digits=\noexpand\xinttheDigits)

  ">>> " means central computing is waiting for input
  "... " means that multi-line input continues.  Use `;' to terminate it.

  Say `&bye' at any time to terminate the session and the TeX run.
}}%
\catcode`?0\catcode`\\12\relax%
?xdef?xs_help_message_i{%
  - Terminate inputs with the `;' character.  Inputs can extend over
    multiple lines (each terminated by <return>).

  - An empty input (`;' terminated) or the magic word `&pause' puts
    the session on hold and switches to normal TeX interaction, with
    its `*' prompt.  Use \xintsession to resume the session.

  - Lines starting with any control sequence are handed over to
    TeX (no end of line space token is included), and the current
    numerical input resumes on next line.

  - The computing kernel can work in various modes:
    exact (uses \xintexpr), fp (uses \xintfloatexpr), int (uses
    \xintiiexpr), and pol (makes definitions using \poldef).
    To activate a mode enter `&<mode name>' at the prompt

  - Issue `&mode' if you forgot in what mode you are currently in.

(1/5) say `n' (or directly <return>) for next, `q' to quit, or one of
      `2', `3', `4', or `5' to jump directly to corresponding panel.
}%
?xdef?xs_help_message_ii{%
  - Say `&exact' resp. `&fp' to go back and forth between exact
    and floating point mode

  - Say `&fp=<number>' for example `&fp=32' at the prompt to activate
    fp mode at a given Digits setting.  This can be done in the midst
    of the input of a multi-line computation!

  - Math functions are supported up to Digits=62 (since xint 1.4e);
    sqrt() and powers with integer exponents work in fp mode with Digits
    arbitrary.

  - Definitions (variables or functions) are done using `:='.

    *  Parentheses present before the `:=' trigger a function
       definition, else a variable definition is done.

    *  IMPORTANT: using a `=' by oversight will cause breakage as
    xintsession thinks it is handling an evaluation, not a definition.

%    Perhaps in future it will be allowed to make definitions using `='
%    and xintsession will check if it is not misinterpreting a `=' from
%    `<=', `>=', `!=', `==' or `seq()' syntax but this currently was
%    considered to add too much overhead for the benefit.
%    
(2/5) say `n' (next), `p' (previous), `q' (quit), or one of `1' to `5'.
}%
?xdef?xs_help_message_iii{%
  - Functions defined in one mode are not available for usage in
    other modes without further steps; check xintexpr doc.

  - Polynomials in pol mode are defined using polname(x):=expression;
    (or with another letter).  They can serve as functions in the exact
    mode.  To use a polynomial also as an fp mode function, issue
    &GenFloat(<polname>) at the prompt.

  - Whenever a variable is defined, or a computation is performed,
    the result is assigned a label @_<integer>.  These labels are
    variables which can then be used in subsequent computations.

  - The special variables @_, @__, and @___ hold the last, next
    to last and next to next to last evaluation results.

  - In the case of a multi-variable assignment (see xintexpr doc)
    the automatic variable @_<integer> holds the whole thing, which
    may be a nutple or an ople, in addition to individual variables.

(3/5) say `n' (next), `p' (previous), `q' (quit), or one of `1' to `5'.
}%
?xdef?xs_help_message_iv{%
  - Variables defined in any mode can be used in any mode except:
    * polynomials can be used as variables only in the pol and the
      exact modes,
    * integer-only mode requires... integer-only variables, also
      internally.  Use num() or \xintiexpr...\relax as wrappers in int
      mode around exact or fp variables (even integer-valued ones).

  Known issues:

  - The [P] optional argument of \xintfloateval can not be used
    because it is not recognized by \xintdeffloatvar, one can use
    the float() function with its second argument.

  - An \xintexpr...\relax wrapper can not start a line, as it would
    trigger control sequence execution.  Use parentheses or 0+\...

  - Exact mode does not reduce to lowest terms, use reduce().

(4/5) say `n' (next), `p' (previous), `q' (quit), or one of `1' to `5'.
}%
?xdef?xs_help_message_v{%
  (cont.)
  - A session transcript file is created which can be executed by
    eTeX to redo all calculations; however @_, @__, and @___ will
    not be defined in this context so need to be manually replaced
    by the suitable @_<integer>.  The format of the transcript
    is to be considered unstable at this stage.

  Future:

  - a plot() function will be added at some point to export
    coordinates into a .tex file for usage either with PSTricks or
    TikZ/pgf.

  Miscellany:

  Some exceptions are caught by xintexpr itself: enter <return> at
  the prompt and a recovery will be attempted.   If deep in core
  TeX errors, try escaping using `S'.  It does work sometimes.

(5/5) say `p' (previous), `q' (quit), or one of `1' to `5'.
}%
?endgroup
\def\xs_help\xs_fetch_aa{\xs_help_i}%
\def\xs_help_i
{%
  \message{\xs_help_message_i}%
  \def\xs_help_next##1\xs_help_next
  {%
   \if\xintFirstItem{##1n}n\xint_dothis\xs_help_ii\fi%
   \ifnum10<1\expandafter\string\romannumeral0\xintfirstitem{##1}\relax%
      \xint_dothis{\csname xs_help_\romannumeral\xintFirstItem{##1}\endcsname}\fi%
   \xint_orthat\xs_help_quit
  }%
  \message{n, q, or 1 to 5: }%
  \read-1to\xs_tmp\expandafter\xs_help_next\xs_tmp\xs_help_next
}%
\def\xs_help_ii
{%
  \message{\xs_help_message_ii}%
  \def\xs_help_next##1\xs_help_next
  {%
   \if\xintFirstItem{##1n}n\xint_dothis\xs_help_iii\fi%
   \if\xintFirstItem{##1}p\xint_dothis\xs_help_i\fi%
   \ifnum10<1\expandafter\string\romannumeral0\xintfirstitem{##1}\relax%
      \xint_dothis{\csname xs_help_\romannumeral\xintFirstItem{##1}\endcsname}\fi%
   \xint_orthat\xs_help_quit
  }%
  \message{n, p, q, or 1 to 5: }%
  \read-1to\xs_tmp\expandafter\xs_help_next\xs_tmp\xs_help_next
}%
\def\xs_help_iii
{%
  \message{\xs_help_message_iii}%
  \def\xs_help_next##1\xs_help_next
  {%
   \if\xintFirstItem{##1n}n\xint_dothis\xs_help_iv\fi%
   \if\xintFirstItem{##1}p\xint_dothis\xs_help_ii\fi%
   \ifnum10<1\expandafter\string\romannumeral0\xintfirstitem{##1}\relax%
      \xint_dothis{\csname xs_help_\romannumeral\xintFirstItem{##1}\endcsname}\fi%
   \xint_orthat\xs_help_quit
  }%
  \message{n, p, q, or 1 to 5: }%
  \read-1to\xs_tmp\expandafter\xs_help_next\xs_tmp\xs_help_next
}%
\def\xs_help_iv
{%
  \message{\xs_help_message_iv}%
  \def\xs_help_next##1\xs_help_next
  {%
   \if\xintFirstItem{##1n}n\xint_dothis\xs_help_v\fi%
   \if\xintFirstItem{##1}p\xint_dothis\xs_help_iii\fi%
   \ifnum10<1\expandafter\string\romannumeral0\xintfirstitem{##1}\relax%
      \xint_dothis{\csname xs_help_\romannumeral\xintFirstItem{##1}\endcsname}\fi%
   \xint_orthat\xs_help_quit
  }%
  \message{n, p, q, or 1 to 5: }%
  \read-1to\xs_tmp\expandafter\xs_help_next\xs_tmp\xs_help_next
}%
\def\xs_help_v
{%
  \message{\xs_help_message_v}%
  \def\xs_help_next##1\xs_help_next
  {%
   \if\xintFirstItem{##1n}p\xint_dothis\xs_help_iv\fi%
   \ifnum10<1\expandafter\string\romannumeral0\xintfirstitem{##1}\relax%
      \xint_dothis{\csname xs_help_\romannumeral\xintFirstItem{##1}\endcsname}\fi%
   \xint_orthat\xs_help_quit
  }%
  \message{p, q, or 1 to 5: }%
  \read-1to\xs_tmp\expandafter\xs_help_next\xs_tmp\xs_help_next
}%
\def\xs_help_quit{\xs_welcome_back\xs_fetch}%
\let\xs_magic@&hel\xs_help\let\xs_help_vi\xs_help_quit\let\xs_help_vii\xs_help_quit
\let\xs_help_viii\xs_help_quit\let\xs_help_ix\xs_help_quit
%
\xsrestorecatcodes%
\csname xs\string _storeenv\endcsname\endlinechar13\relax%
% extremely eTeX-weird bug of 0.3 which had here \endinput\xintsession,
% causing breakage if the latter attempts a \scantokens in an \edef
% (which is what happens for function definitions).
% bugfix 0.3a by simply moving \xintsession to before the \endinput
\xintsession\endinput%
